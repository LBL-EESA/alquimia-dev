\documentclass{beamer}

\usepackage{comment}
\usepackage{color}
\usepackage{listings}
\usepackage{verbatim}
\usepackage{multicol}
\usepackage{booktabs}
\usepackage{xspace}
\usepackage{hyperref}

\hypersetup{colorlinks=true,citecolor=blue,filecolor=blue,linkcolor=blue,
urlcolor=blue,breaklinks=true}

\definecolor{green}{RGB}{0,128,0}

\def\EQ#1\EN{\begin{equation*}#1\end{equation*}}
\def\BA#1\EA{\begin{align*}#1\end{align*}}
\def\BS#1\ES{\begin{split*}#1\end{split*}}
\newcommand{\bc}{\begin{center}}
\newcommand{\ec}{\end{center}}
\newcommand{\eq}{\ =\ }
\newcommand{\degc}{$^\circ$C}

\def\p{\partial}
\def\qbs{\boldsymbol{q}}
\def\Dbs{\boldsymbol{D}}
\def\A{\mathcal A}
\def\gC{\mathcal C}
\def\gD{\mathcal D}
\def\gL{\mathcal L}
\def\M{\mathcal M}
\def\P{\mathcal P}
\def\Q{\mathcal Q}
\def\gR{\mathcal R}
\def\gS{\mathcal S}
\def\X{\mathcal X}
\def\bnabla{\boldsymbol{\nabla}}
\def\bnu{\boldsymbol{\nu}}
\renewcommand{\a}{{\alpha}}
%\renewcommand{\a}{{}}
\newcommand{\s}{{\sigma}}
\newcommand{\bq}{\boldsymbol{q}}
\newcommand{\bz}{\boldsymbol{z}}
\def\bPsi{\boldsymbol{\Psi}}

\def\Li{\textit{L}}
\def\Fb{\textbf{f}}
\def\Jb{\textbf{J}}
\def\cb{\textbf{c}}

\def\Dt{\Delta t}
\def\tpdt{{t + \Delta t}}
\def\bpsi{\boldsymbol{\psi}}
\def\dbpsi{\delta \boldsymbol{\psi}}
\def\bc{\textbf{c}}
\def\dbc{\delta \textbf{c}}
\def\arrows{\rightleftharpoons}

\newcommand{\bGamma}{\boldsymbol{\Gamma}}
\newcommand{\bOmega}{\boldsymbol{\Omega}}
%\newcommand{\bPsi}{\boldsymbol{\Psi}}
%\newcommand{\bpsi}{\boldsymbol{\psi}}
\newcommand{\bO}{\boldsymbol{O}}
%\newcommand{\bnu}{\boldsymbol{\nu}}
\newcommand{\bdS}{\boldsymbol{dS}}
\newcommand{\bg}{\boldsymbol{g}}
\newcommand{\bk}{\boldsymbol{k}}
%\newcommand{\bq}{\boldsymbol{q}}
\newcommand{\br}{\boldsymbol{r}}
\newcommand{\bR}{\boldsymbol{R}}
\newcommand{\bS}{\boldsymbol{S}}
\newcommand{\bu}{\boldsymbol{u}}
\newcommand{\bv}{\boldsymbol{v}}
%\newcommand{\bz}{\boldsymbol{z}}
\newcommand{\pressure}{P}

\def\water{H$_2$O}
\def\calcium{Ca$^{2+}$}
\def\copper{Cu$^{2+}$}
\def\magnesium{Mg$^{2+}$}
\def\sodium{Na$^+$}
\def\potassium{K$^+$}
\def\uranium{UO$_2^{2+}$}
\def\hion{H$^+$}
\def\hydroxide{0H$^-$}
\def\bicarbonate{HCO$_3^-$}
\def\carbonate{CO$_3^{2-}$}
\def\cotwo{CO$_2$(aq)}
\def\chloride{Cl$^-$}
\def\fluoride{F$^-$}
\def\phosphoricacid{HPO$_4^{2-}$}
\def\nitrate{NO$_3^-$}
\def\sulfate{SO$_4^{2-}$}
\def\souotwooh{$>$SOUO$_2$OH}
\def\sohuotwocothree{$>$SOHUO$_2$CO$_3$}
\def\soh{$>$SOH}

\newcommand{\pft}{PFLOTRAN\xspace}

\newcommand\add[1]{{{\color{blue} #1}}}
\newcommand\remove[1]{\sout{{\color{red} #1}}}
\newcommand\codecomment[1]{{{\color{green} #1}}}
\newcommand\redcolor[1]{{{\color{red} #1}}}
\newcommand\bluecolor[1]{{{\color{blue} #1}}}
\newcommand\greencolor[1]{{{\color{green} #1}}}
\newcommand\magentacolor[1]{{{\color{magenta} #1}}}
\newcommand\gehcomment[1]{{{\color{orange} #1}}}

\def\aligntop#1{\vtop{\null\hbox{#1}}}

\begin{comment}
\tiny
\scriptsize
\footnotesize
\small
\normalsize
\large
\Large
\LARGE
\huge
\Huge
\end{comment}

%\usetheme[height=7mm]{Rochester}  % No navigation bar
\setbeamertemplate{blocks}[rounded][shadow=true]

\setbeamersize{text margin left=4mm, text margin right=4mm}
  % Do this so there isn't so much white space before bullet items! --RTM

\beamertemplatenavigationsymbolsempty % to get rid of nav symbols
%\setbeamertemplate{frames}{}


\begin{document}
\title[Alquimia]{Alquimia: an API for ASCEM geochemistry}
\author[]{Ben Andre, Glenn Hammond, Carl Steefel}
\date{\today}

%%%%%%%%%%%%%%%%%%%%%%%%%%%%%%%%%%%%%%%%%%%%%%%%%%%%%%%%%%%%%%%%%%%%%%%%%%%%%%%
\begin{frame}{Template}
\end{frame}
%%%%%%%%%%%%%%%%%%%%%%%%%%%%%%%%%%%%%%%%%%%%%%%%%%%%%%%%%%%%%%%%%%%%%%%%%%%%%%%

%%%%%%%%%%%%%%%%%%%%%%%%%%%%%%%%%%%%%%%%%%%%%%%%%%%%%%%%%%%%%%%%%%%%%%%%%%%%%%%
\frame{\titlepage}
%\section{Introduction}
%%%%%%%%%%%%%%%%%%%%%%%%%%%%%%%%%%%%%%%%%%%%%%%%%%%%%%%%%%%%%%%%%%%%%%%%%%%%%%%

%%%%%%%%%%%%%%%%%%%%%%%%%%%%%%%%%%%%%%%%%%%%%%%%%%%%%%%%%%%%%%%%%%%%%%%%%%%%%%%
\begin{frame}{The Alquimia Philosophy}
\Large
\begin{itemize}
\item What Alquimia \bluecolor{is}:
\begin{itemize}
\Large
\item A wrapper facilitating interfacing with existing, external
      geochemistry codes as 3$^\text{rd}$ party libraries
\item Geochemical democracy
\end{itemize}

\vspace{1cm}
\item What Alquimia \redcolor{is not}:
\begin{itemize}
\Large
\item A geochemistry library implementation
\item Geochemical coercion
\end{itemize}
\end{itemize}
\end{frame}
%%%%%%%%%%%%%%%%%%%%%%%%%%%%%%%%%%%%%%%%%%%%%%%%%%%%%%%%%%%%%%%%%%%%%%%%%%%%%%%

%%%%%%%%%%%%%%%%%%%%%%%%%%%%%%%%%%%%%%%%%%%%%%%%%%%%%%%%%%%%%%%%%%%%%%%%%%%%%%%
\begin{frame}{Alquimia Concept}
Provide a uniform application programming interface (API) through which (solute) transport simulators may couple to third-party geochemical reaction libraries.  The API
\begin{itemize}
\item Provides a common set of primitive data structures to which data structures from 3$^\text{rd}$ party libraries may be mapped.
\item Provides common handles (routines) for
\begin{itemize}
\item Reading and initializing the geochemical basis and associated reactions.
\item Integrating a time step at a single grid cell.
\item Extracting geochemical concentrations.
\end{itemize}
\item Serves solely to pass information between transport and reaction components (i.e. it does not drive the geochemical solution procedure).
\end{itemize}
\end{frame}
%%%%%%%%%%%%%%%%%%%%%%%%%%%%%%%%%%%%%%%%%%%%%%%%%%%%%%%%%%%%%%%%%%%%%%%%%%%%%%%

%%%%%%%%%%%%%%%%%%%%%%%%%%%%%%%%%%%%%%%%%%%%%%%%%%%%%%%%%%%%%%%%%%%%%%%%%%%%%%%
\begin{frame}[allowframebreaks]{Timeline (FY13)}
\begin{itemize}
\item Alquimia interface
\begin{itemize}
\item Preliminary API - Nov
\item Batch chemistry driver - Jan
\item Process constraints with database - Feb
\item Reactive transport driver - Mar
\item Amanzi driven reactive transport - Jun
\end{itemize}
\item Geochemistry database
\begin{itemize}
\item With unexpected Akuna deliverables, no commitment until FY 14
\end{itemize}

\newpage

\item Amanzi chemistry library
\begin{itemize}
\item Reaction step through Alquimia - Jan
\item \gehcomment{Do we really want to support a non-native interface?}
\item \gehcomment{Who is responsible for Amanzi chemistry maintenance?}
\end{itemize}
\item CrunchFlow
\begin{itemize}
\item Bullets on implementation....
\item Timeline to parallel \pft
\end{itemize}
\item \pft
\begin{itemize}
\item Regression test suite - Oct
\item Isolation of \pft chemistry - Oct
\item \pft chemsitry
\begin{itemize}
\item Reaction step - Dec
\item Basis Management - Jan
\item Interface to new geochemical database - FY14
\end{itemize}
\end{itemize}
\end{itemize}
\end{frame}
%%%%%%%%%%%%%%%%%%%%%%%%%%%%%%%%%%%%%%%%%%%%%%%%%%%%%%%%%%%%%%%%%%%%%%%%%%%%%%%

%%%%%%%%%%%%%%%%%%%%%%%%%%%%%%%%%%%%%%%%%%%%%%%%%%%%%%%%%%%%%%%%%%%%%%%%%%%%%%%
\begin{frame}{Alquimia: Required Functionality}
\begin{itemize}
\item Handles for
\begin{itemize}
\item Reading of geochemical reaction data
\item Basis management (e.g. reading and swapping the basis)
\item Constraint processing for geochemical conditions (e.g. boundary/initial conditions, source/sinks) with respect to the desired basis
\item Speciation
\item Reaction stepping in operator split mode
\item Evaluation of local residual and Jacobian entries for grid cell
\begin{itemize}
\item Note that \emph{evaluation} does not construct the global Jacobian
\end{itemize}
\item Delivery of secondary variables (e.g. pH, reaction rates, mineral saturation, etc.) at the request of the client
\end{itemize}
\end{itemize}
\end{frame}
%%%%%%%%%%%%%%%%%%%%%%%%%%%%%%%%%%%%%%%%%%%%%%%%%%%%%%%%%%%%%%%%%%%%%%%%%%%%%%%

%%%%%%%%%%%%%%%%%%%%%%%%%%%%%%%%%%%%%%%%%%%%%%%%%%%%%%%%%%%%%%%%%%%%%%%%%%%%%%%
\begin{frame}{Division of Labor}
\begin{tabular}{c|c|c}
\aligntop{
\begin{minipage}{0.3\textwidth}
\textbf{Client}
\begin{itemize}
\small
\item Process kernel implemented by developers of the transport code
\item Manage global storage
\item Loop through space
\item Unpack and move data from mesh dependent storage into Alquimia data transfer containers
\item Manage time stepping, sub-stepping, error handling etc.
\end{itemize}
\end{minipage}} &
\aligntop{
\begin{minipage}{0.3\textwidth}
\textbf{Alquimia wrapper}
\begin{itemize}
\small
\item Defines an engine independent API
\item Provides data munging or delivers raw data
\item \redcolor{No} geochemical calculations
\end{itemize}
\end{minipage}} &
\aligntop{
\begin{minipage}{0.3\textwidth}
\textbf{Engine}
\begin{itemize}
\small
\item Provides all geochemical functionality, i.e. basis
  management, constraint processing, reaction stepping (OS or GI)
\end{itemize}
\end{minipage}} \\
\end{tabular}
\end{frame}
%%%%%%%%%%%%%%%%%%%%%%%%%%%%%%%%%%%%%%%%%%%%%%%%%%%%%%%%%%%%%%%%%%%%%%%%%%%%%%%
%%%%%%%%%%%%%%%%%%%%%%%%%%%%%%%%%%%%%%%%%%%%%%%%%%%%%%%%%%%%%%%%%%%%%%%%%%%%%%%
\begin{frame}{Constraint Processing}
\begin{itemize}
\item Domain scientist rarely want to define the IC/BC for a problem
  purely in terms of total component concentrations.
\item Often want to specify the system in variables that are
  closer to observational data. for example:
  \begin{itemize}
    \item pH instead of Total $H^+$
    \item aqueous $CO_2$ at equilibrium with the atmosphere
    \item $Ca^{2+}$ in equilibrium with calcite
    \item $Cl^-$ based on charge balance.
  \end{itemize}
\item What does this mean for state, transport and input processing?
\item Initial and boundary conditions are no longer simple lists of
  dirichlet conditions.
\item ``Geochemical Conditions'' must be preprocessed by the
  geochemistry engine to determine the appropriate dirichlet condition
  to apply.
\item ...
\end{itemize}
\end{frame}
%%%%%%%%%%%%%%%%%%%%%%%%%%%%%%%%%%%%%%%%%%%%%%%%%%%%%%%%%%%%%%%%%%%%%%%%%%%%%%%

\end{document}
